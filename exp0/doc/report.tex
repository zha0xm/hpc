\documentclass[a4paper, 12pt]{article}
\usepackage[utf8]{inputenc}
\usepackage[T1]{fontenc}
\usepackage{ctex}
\usepackage{geometry}
\usepackage{amsmath}
\usepackage{amssymb}
\usepackage{graphicx}
\usepackage{listings}
\usepackage{xcolor}
\usepackage{hyperref}
\usepackage{float}


\lstset{
  language=C++,
  basicstyle=\ttfamily\footnotesize,
  keywordstyle=\color{blue},
  commentstyle=\color{green!50!black},
  stringstyle=\color{red},
  breaklines=true,
  numbers=left,
  numberstyle=\tiny\color{gray},
  frame=single
}

\title{实验报告: MPI 和 OpenMP 并行化 pow\_a}
\author{赵袭明 \\ 2021012319}
\date{\today}

\begin{document}

\maketitle

\section{实验任务}
使用 MPI 和 OpenMP 并行化下述代码,代码的作用是计算 $b[i] = a[i]^m$ 其中 $a[i]$ 和 $b[i]$ 是两个长度为 $n$ 的数组。
\begin{lstlisting}
void pow_a(int *a, int *b, int n, int m) {
    for (int i = 0; i < n; i++) {
        int x = 1;
        for (int j = 0; j < m; j++)
            x *= a[i];
        b[i] = x;
    }
}
\end{lstlisting}

\section{代码实现}

\subsection{OpenMP 版本}
修改后的 \texttt{pow\_a} 函数如下:

\begin{lstlisting}
#include <omp.h>
void pow_a(int *a, int *b, int n, int m) {
    #pragma omp parallel for
    for (int i = 0; i < n; i++) {
        int x = 1;
        for (int j = 0; j < m; j++)
            x *= a[i];
        b[i] = x;
    }
}
\end{lstlisting}

只需加入一行指导语句即可。

\subsection{MPI 版本}
修改后的 \texttt{pow\_a} 函数如下:
\begin{lstlisting}
#include <mpi.h>
void pow_a(int *a, int *b, int n, int m, int comm_sz) {
    int local_n = n / comm_sz;
    for (int i = 0; i < local_n; i++) {
        int x = 1;
        for (int j = 0; j < m; j++)
            x *= a[i];
        b[i] = x;
    }
}
\end{lstlisting}

在 \texttt{pow\_a} 函数内, 每个进程只计算属于自己的 $a$ 数组部分,对 $a[i]$ 进行 $m$ 次方运算并存储到 $b[i]$即可。

\section{实验结果}
原始的代码输出如下:
\begin{verbatim}
g++ openmp_pow.cpp -O3 -std=c++11 -fopenmp -o openmp_pow
mpicxx mpi_pow.cpp -O3 -std=c++11 -o mpi_pow
openmp_pow: n = 112000, m = 100000, thread_count = 1
Congratulations!
Time Cost: 14015005 us

openmp_pow: n = 112000, m = 100000, thread_count = 7
Congratulations!
Time Cost: 14018343 us

openmp_pow: n = 112000, m = 100000, thread_count = 14
Congratulations!
Time Cost: 14008828 us

openmp_pow: n = 112000, m = 100000, thread_count = 28
Congratulations!
Time Cost: 14009171 us

mpi_pow: n = 112000, m = 100000, process_count = 1
Wrong answer at position 34133: 0 != -259604863
srun: error: conv1: task 0: Exited with exit code 1
\end{verbatim}

\subsection{OpenMP 版本实验结果}
实验环境:
$n = 112000, m = 100000$


使用 OpenMP 和 MPI 修改后的到了如下输出:
\begin{verbatim}
g++ openmp_pow.cpp -O3 -std=c++11 -fopenmp -o openmp_pow
mpicxx mpi_pow.cpp -O3 -std=c++11 -o mpi_pow
openmp_pow: n = 112000, m = 100000, thread_count = 1
Congratulations!
Time Cost: 14011480 us

openmp_pow: n = 112000, m = 100000, thread_count = 7
Congratulations!
Time Cost: 2013215 us

openmp_pow: n = 112000, m = 100000, thread_count = 14
Congratulations!
Time Cost: 1010784 us

openmp_pow: n = 112000, m = 100000, thread_count = 28
Congratulations!
Time Cost: 516337 us

mpi_pow: n = 112000, m = 100000, process_count = 1
Congratulations!
Time Cost: 14009948 us

mpi_pow: n = 112000, m = 100000, process_count = 7
Congratulations!
Time Cost: 2010744 us

mpi_pow: n = 112000, m = 100000, process_count = 14
Congratulations!
Time Cost: 1005803 us

mpi_pow: n = 112000, m = 100000, process_count = 28
Congratulations!
Time Cost: 501768 us

mpi_pow: n = 112000, m = 100000, process_count = 56
Congratulations!
Time Cost: 370754 us

All done!
\end{verbatim}

\section{运行时间比较}
\begin{table}[h]
    \centering
    \begin{tabular}{|c|c|c|}
        \hline
        线程数 & 运行时间(us) & 加速比  \\
        \hline
        1 & 14009948 & 1    \\
        7 & 2013215 & 6.96  \\
        14 & 1010784 & 13.86    \\
        28 & 516337 & 27.14 \\
        \hline
    \end{tabular}
    \caption{OpenMP 实验结果}
\end{table}
可以看到基本实现了随线程数的线性加速

\begin{table}[h]
    \centering
    \begin{tabular}{|c|c|c|}
    \hline
    $N \times P$ & 运行时间(us) & 加速比  \\
    \hline
    $1 \times 1$ & 14011480 & 1    \\
    $1 \times 7$ & 2010744 & 6.97  \\
    $1 \times 14$ & 1005803 & 13.93    \\
    $1 \times 28$ & 501768 & 27.92 \\
    $2 \times 28$ & 370754 & 37.79  \\
    \hline
    \end{tabular}
    \caption{MPI 实验结果}
\end{table}
MPI 的实验结果在机器数 $N = 1$ 时基本实现了线性的加速比,
在 $N = 2$ 时加速比偏离了线性, 可能是因为机器间的通信消耗了较多的时间。


\end{document}